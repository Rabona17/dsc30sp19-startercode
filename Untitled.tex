\documentclass[14pt]{extarticle}
\usepackage{geometry}
\usepackage{amsfonts}
 \geometry{
 a4paper,
 total={170mm,257mm},
 left=20mm,
 top=20mm,
 }

\title{Math109 Week2 Lecture 1}
\begin{document}
\maketitle{}
\section{Proof by contradiction}
{\bfseries Example 1:}
\begin{flushleft}
\vspace{0.3cm}Prove that
\end{flushleft}

\vspace{0.3cm}\begin{equation}
78m+102n=11
\end{equation}
doesn't have an integer solution.
\begin{flushleft}
{\bfseries Solution:}
\end{flushleft}
\vspace{0.3cm}Assume by contradiction that there is an integer solution, i.e. that for some $m,n \in \mathbb{Z}, 78m+102n=1$. Since 2 divides 78 and 102, $78m+12n$ is an even number, but 11 is an odd number, a contradiction.
\begin{flushleft}
{\bfseries Example 2:}
\vspace{0.3cm}Prove that
\end{flushleft}
\vspace{0.3cm}\begin{equation}
\sqrt{2}\; is \; irrational.
\end{equation}
\vspace{0.3cm}x is rational if there exists $p,q \in  \mathbb{Z}, q \neq 0$, such that $x = \frac{p}{q}$. x is called irrational if it is not rational.
\begin{flushleft}
{\bfseries Solution:} 
\end{flushleft}
\vspace{0.3cm}Assume by contradiction that there exists $p,q \in \mathbb{Z}, q \neq0$, so that $\sqrt{2} = \frac{p}{q}$  and $\frac{p}{q}$ is an irreducible fraction. 

\vspace{0.3cm}Then:
\begin{equation}
\sqrt{2}q=p
\end{equation}
\vspace{0.3cm}\begin{equation}
2q^{2}=p^{2} \; \Rightarrow \;p^{2} \;is\; even\; \Rightarrow\; p \;is\;even.
\end{equation}
\vspace{0.3cm}(to be proved). Then there exists $k \in \mathbb{Z}$ so that 
\begin{equation}
p=2k \Rightarrow 2q^{2} = 4k^{2} \Rightarrow q\;is\;even.
\end{equation}
Thus, since $q,p$ are both even, they are reducible, which is an contradiction.
\begin{flushleft}
\vspace{0.3cm}Now let's prove $p^{2} \;is\; even\; \Rightarrow\; p \;is\;even$:

\vspace{0.3cm}...
\end{flushleft}
\end{document}